

From Table \ref{tab:IPHCdata}, four issues are apparent: 

\begin{resdoclist}

\item For 1997-2002 and 2013 only the first 20~hooks of each skate were
enumerated, whereas for all other years all hooks were enumerated. Thus, the
data from each year cannot simply be considered as comparable and analysed as
one consecutive time series.

\item For the datasets for 1995, 1996, 1997-2002 and 2013, data are only
available at the set-by-set level, in terms of numbers of a given species per
effective skate. Which species was caught on each hook is not available, unlike
for 2003-2012 and 2014. Thus, for 1995 and 1996 we cannot calculate catch rates
based on the first 20~hooks (because we only have set-by-set level data),
whereas we can do that for 2003-2012 and 2014, and the 20-hook data is the only
information we have for 1997-2002 and 2013.

\item In 2012 a bait experiment was conducted such that data from all skates
could not be used; see Section \ref{sec:chum}.

\item The WCVI was not visited in every year, so the spatial coverage is not
consistent across years.

\end{resdoclist}


To address issues 1, 2 and 4 we therefore construct four time series (whose
structure is summarised in Table \ref{tab:seriesSumm}):

\begin{table}[t] \centering
\caption{Summary of how the four Series {\bf A}, {\bf B}, {\bf C} and {\bf D}
are constructed. Numbers in parentheses indicate the number of years for which
data for each Series are available. `Only north of WCVI' indicates Series that
only consider stations north of Vancouver Island (thus excluding those off the
WCVI), `Full coast' indicates Series that use all stations from the whole
coast. The rows indicate how many hooks the catch rates for each Series are
based on.}
\label{tab:seriesSumm}
\begin{tabular}{rcc}
\hline
 & Only north of WCVI & Full coast \\
\hline
First 20 hooks from each skate & {\bf A}(18) & {\bf D}(15)\\
All hooks from each skate & {\bf B}(13) & {\bf C}(11)\\
\hline
~\\    % to add some whitespace before text carries on
\end{tabular}
\end{table}

{\bf Series A} -- 1997-2014 stations north of WCVI, with catch rates based on
first 20~hooks only (which is all we have for 1997-2002 and 2013).  % 1997-2002,
2003-2012, 2013 and 2014
 
{\bf Series B} -- 1995, 1996, 2003-2012 and 2014 stations north of WCVI, with
catch rates based on all hooks (which is all we have for 1995 and 1996).

{\bf Series C} -- 2003-2012 and 2014 stations coastwide (including WCVI), with
catch rates based on all hooks.

{\bf Series D} -- 1999 and 2001-2014 stations coastwide (including WCVI), with
catch rates based on first 20~hooks only (which is all we have for 1999,
2001-2002 and 2013).

We would like to obtain an index series with as long a timespan as possible,
and, ideally, over as broad a geographic region as possible. Since Series~A is
the longest time series, we take this and expand it to Series~AB, defined as:

{\bf Series AB} -- for stations north of WCVI, combine the 1995 and 1996 values
from Series B, based on all hooks, with the 1997-2014 values from Series A that
are based on first 20 hooks only. See Section~\ref{sec:combine}.

The resulting Series~AB covers the stations north of WCVI. In
Sections~\ref{sec:compareAD} and~\ref{sec:compareBC} we show why we can consider
this series to be representative of the full coast (i.e.~including the WCVI), by
comparing the series that exclude stations off the WCVI (Series~A and~B) with
those that include the stations off the WCVI (Series~D and~B, respectively).

The absolute catch rate index for Series~AB could therefore be justified as
being an input to an assessment model, and be considered to be an index for the
whole coast of British Columbia. Though recall that the model for this stock
assessment uses the output from Appendix~C, rather than from this Appendix.

\section{SPATIAL LOCATIONS OF STATIONS}

The maps in Figures \ref{fig:stations1995mapA}-\ref{fig:stations2013mapAkeep}
show the locations of the stations of the IPHC survey since 1995. Early on,
stations were not fixed between years, with the main difference being whether or
not the waters off the west coast of Vancouver Island were surveyed (as
summarised in Table \ref{tab:seriesSumm}).

From 1995-1997 (Figures
\ref{fig:stations1995mapA}-\ref{fig:stations1997mapAkeep}) stations were
arranged in Y-shapes; they were not exactly the same locations each year, but
fairly close to each other. From 1998 onwards (Figures
\ref{fig:stations1998mapAkeep}-\ref{fig:stations2013mapAkeep}) the stations have
been positioned equidistant from one another on a 10-nautical-mile square grid
\citep{fycd12}. In 1999 (Figure \ref{fig:stations1999mapAkeep}) the survey first
went to the WCVI. From 2001 onwards, the survey was consistently conducted at
170 regular fixed (non-random) stations (Figures
\ref{fig:stations2001mapAkeep}-\ref{fig:stations2013mapAkeep}).

Given the difference in coverage between years, for Series~A and~B we exclude
those stations south of 50.6$^{\circ}$ latitude, which is near the northern tip
of Vancouver Island (black line in Figure~\ref{fig:stations1995mapA}). This
latitude was chosen so that all the stations from 1995-1997 are included
(Figures~\ref{fig:stations1995mapA}-\ref{fig:stations1997mapAkeep}). The
stations for 1995-1997 show good overlap (north of Vancouver Island) with the
stations from 1998 onwards
(Figures~\ref{fig:stations1998mapAkeep}-\ref{fig:stations2013mapAkeep}).

For 1999 (Figure~\ref{fig:stations1999mapAkeep}) and 2001 onwards
(Figures~\ref{fig:stations2001mapAkeep}-\ref{fig:stations2013mapAkeep}), the
black crosses indicate the stations off the WCVI that are below 50.6$^{\circ}$
latitude and are therefore excluded from Series~A and~B. Series~C and~D use all
stations coastwide.

For Series A~and~D we only consider the first 20 hooks from each skate. For
2003-2012 and 2014 we have data for all hooks, and so in
Figure~\ref{fig:allBlocks0314map} we illustrate the stations where a \spName~was
never caught (in any year from 2003-2012 and 2014) on any hook, as well as the
stations that caught \spName~in some years but never in the first 20 hooks, and
the stations that did catch it in the first 20 hooks of each skate (for at least
one year).

% From iphcSerBallHooksYYR.tex
% \onefig{stations1995mapA}{Locations of the 120 stations in 1995, of which 81 did not catch Yelloweye Rockfish (red open circles), 34 stations did catch it (red closed circles), and 5 were deemed unusable by the IPHC (grey closed circles) and so are not considered further. The black line indicates the geographic cut-off, below which stations are excluded when constructing Series~A and~B; for this year no stations (black crosses) are excluded, since the cut-off was chosen so as to include all stations for 1995, 1996 and 1997.}{stations1995mapA}{5}

\section{CHUM SALMON BAIT EXPERIMENT}\label{sec:chum}

Prior to 2012, Chum Salmon (\emph{Oncorhynchus keta}) was used for bait. But in
2012, a bait experiment was conducted \citep{hsdgr13}. At each station three
different bait types were used on the same set: a consecutive four-skate Chum
Salmon treatment, a one-skate Pink Salmon (\emph{Oncorhynchus gorbuscha})
treatment, and a one-skate Walleye Pollock (\emph{Theragra chalcogramma})
treatment. The location of the three treatments on each set was randomized
throughout the survey, and each treatment was separated by one skate (1,800 ft)
of hookless groundline. For consistency with previous years, we only consider
the four skates that used Chum Salmon as bait.

The effective skate number provided by the IPHC is for all skates used, which in
2012 will include skates that were not baited with Chum Salmon (Eric Soderlund,
IPHC, Seattle, WA, USA, pers.~comm.). But we wish to only include the Chum
Salmon baited skates, and so we need to modify the effective skate number (see
below). The effective skate number depends on the number of observed hooks (Eric
Soderlund, IPHC, Seattle, WA, USA, pers.~comm.), rather than the number of hooks
that were deployed. The bait experiment was not continued for 2013 or 2014.

% \clearpage

\section{CATCH RATE EQUATIONS}\label{sec:notation}

\subsection{CATCH RATE BASED ON ALL CHUM-BAIT HOOKS}

We wish to obtain a catch rate index which, for each year, will be the mean
catch rate across all sets that year. The units will be numbers of
\spName~caught per effective skate. We only want to consider hooks that used
Chum Salmon as bait (hereafter `chum-bait hooks'), because we have no
information as to how catch rates of \spName~may change depending on the bait
used. For our data, 2012 was the only year that hooks were not exclusively
chum-bait hooks.

Define:

$H_{it}$ -- number of observed chum-bait hooks in set $i$ in year
$t$, % {\tt H\_itObsChum}$=${\tt H\_it} later,

$H_{it}^*$ -- number of observed hooks for all bait types
($H_{it} \neq H_{it}^*$ only for 2012), % {\tt H\_itObs},

$E_{it}$ -- effective skate number of set $i$ in year $t$, which needs to be
based on observed chum-bait
hooks, %so is {\tt effSkateChum}$=${\tt E\_it} later.

$E_{it}'$ -- effective skate number from IPHC, which is based on all observed
hooks (regardless of bait). % {\tt effSkate} .

Thus, $E_{it}$ is
\eb
E_{it} = \dfrac{H_{it}}{H_{it}^*} E_{it}'.
\ee

Adapting equations on page 3 of \citet{yocld08}, define:

$N_{it}$ -- the number of fish of a given species caught on set $i=1,2,...,n_t$
in year $t$, based on observed chum-bait hooks, % {\tt N\_itChum}$=${\tt N\_it}.

$n_t$ -- the number of sets in year
$t$, % only including sets which ever catch the species (so not 170 for all years)

$C_{it}$ -- catch rate (with units of numbers per effective skate) of
\spName~for set $i$ in year $t$, based on observed chum-bait hooks, given by
\eb
C_{it} = \frac{N_{it}}{E_{it}}.
\label{catchPerSet}
\ee
The catch rate index for year $t$, $I_t$ (numbers per effective skate), is then
the mean catch rate across all sets:
\eb
I_{t} = \frac{1}{n_t} \sum_{i=1}^{n_t} C_{it} = \frac{1}{n_t} \sum_{i=1}^{n_t} \frac{N_{it}}{E_{it}}.
\label{index}
\ee

\subsection{CATCH RATE BASED THE FIRST 20 CHUM-BAIT HOOKS OF EACH SKATE}

Let $\tilde{X}$ indicate a calculation of some value $X$ that is based only on
the first 20 hooks of each skate. These are the first 20 \emph{numbered} hooks,
not the first 20 \emph{observed} hooks (so not all of the numbered hooks may
have been observed). Thus we have: % For R variables, appending 20 to the
variable names. So notation is:

$\tilde{H}_{it}$ -- number of observed chum-bait hooks in the first 20 hooks of
all skates in set $i$ in year $t$, % {\tt H\_itObsChum20}$=${\tt H\_it20}
later. **May not have available easily and so set to $20 K_{it}$, which is
probably pretty good since snarls etc. happen later in skates.

$\tilde{E}_{it}$ -- effective skate number of set $i$ in year $t$ based on the
first 20 chum-bait hooks that were sent out on each skate. % {\tt
effSkateChum20}$=${\tt E\_it20}.

Since effective skate number is a linear function of the number of hooks in a
set \citep{yocld08}, we have
\eb
\tilde{E}_{it} = \dfrac{\tilde{H}_{it}}{H_{it}} E_{it} \left( = \dfrac{\tilde{H}_{it}}{H_{it}^*} E_{it}' \right).
\label{effSkateScale}
\ee

The resulting notation for the index will be:

$\tilde{I}_{t}$ -- catch rate index for year $t$ (in numbers of \spName~per
effective skate) based on only the first 20 hooks sent out for each skate, %{\tt
I\_tg20},

$\tilde{N}_{it}$ -- the number of \spName~caught on set $i=1,2,...,n_t$ in year
$t$, based on observed chum-bait hooks and only the first 20 hooks sent out for
each skate, % {\tt N\_itChum20}$=${\tt N\_it20} later.

$\tilde{C}_{it}$ -- catch rate (with units of numbers per effective skate) for
set $i$ in year $t$, based only on the first 20 hooks of each skate (and only
skates with chum as bait), such that % {\tt C\_it20}, \eb \tilde{C}_{it} =
\frac{\tilde{N}_{it}}{\tilde{E}_{it}}.
\label{catchPerSet20} \ee The catch rate index for year $t$,
$\tilde{I}_{t\survey}$ (in units of numbers per effective skate), based on only
the first 20 hooks of each skate, is then the mean catch rate across all sets:
\eb
\tilde{I}_{t\survey} = \frac{1}{\tilde{n}_t} \sum_{i=1}^{\tilde{n}_t} \tilde{C}_{it} = \frac{1}{\tilde{n}_t} \sum_{i=1}^{\tilde{n}_t} \frac{\tilde{N}_{it}}{\tilde{E}_{it}}.
% Trying \raisebox, no since that needs $..$ I think:
% \tilde{I}_{t\survey} = \frac{1}{\tilde{n}_t} \raisebox{\depth}{\sum_{i=1}^{\tilde{n}_t}} \tilde{C}_{it} = \frac{1}{\tilde{n}_t} \sum_{i=1}^{\tilde{n}_t} \frac{\tilde{N}_{it}}{\tilde{E}_{it}}.
% Tried this to look better with \sum, but needs work. Leave as is.
% \tilde{I}_{t\survey} = \frac{1}{\tilde{n}_t} \underset{i=1}{\overset{\tilde{n}_t}{\sum}} \tilde{C}_{it} = \frac{1}{\tilde{n}_t} \sum_{i=1}^{\tilde{n}_t} \frac{\tilde{N}_{it}}{\tilde{E}_{it}}.
\label{index20}
\ee
  
\subsection{EQUIVALENCY OF CATCH RATES BASED ON ALL HOOKS AND ON JUST THE FIRST 20 HOOKS}

Equation (\ref{catchPerSet20}) can be written as 
\eb
\tilde{C}_{it} = \frac{\tilde{N}_{it}}{\tilde{E}_{it}} =  \frac{H_{it}}{\tilde{H}_{it}} \frac{\tilde{N}_{it}}{E_{it}}.
\label{tildeCatchPerSet}
\ee
If all hooks are equally likely to catch a \spName, then the catch rates based
on the first 20 hooks of each skate should be an unbiased sample of the catch
rates based on all the hooks.
%
% If the catch rates on the first 20 hooks of each skate are a perfect sample of 
%
The ratio of fish caught, $\tilde{N}_{it} / N_{it}$, should equal (on average)
the ratio of hook numbers, $\tilde{H}_{it} / H_{it}$, because a proportionally
reduced number of fish are caught on the proportionally fewer hooks. Thus
\eb
\dfrac{\tilde{H}_{it}}{H_{it}} = \dfrac{\tilde{N}_{it}}{N_{it}}
\ee
such that
\eb
\tilde{C}_{it} = \dfrac{N_{it}}{\tilde{N}_{it}} \frac{\tilde{N}_{it}}{E_{it}} = \frac{N_{it}}{E_{it}} = C_{it}.
\ee
If the catch rates are greatly different, then this suggests that the catch
rates from the first 20~hooks are not equivalent to the catch rates based on all
the hooks. This is why we compare Series~A and Series~B below.

\section{RESULTS}\label{sec:combine}

\subsection{DETAILS OF SERIES~A AND SERIES~B}

Tables \ref{tab:effSummA} and \ref{tab:effSummB} show the effective skate
numbers for Series~A and~B. The values are lower for Series~A because they are
only based on 20~hooks per skate, compared to all skates for Series~B (see
equation \ref{effSkateScale}). 
%
% Think could have taken both tables straight from iphcCombineAB-YYR.tex
% From iphcSerA20hooksYYR.tex, manually fixing some of the text from the RBR
%  .tex write up.
% latex table generated in R 3.1.0 by xtable 1.7-3 package
% Thu Mar 05 15:22:59 2015
\begin{table}[p]
\centering
\caption{For series A, summary of effective skate numbers, 
     $E_{it}$, for each
     year. Lower and Higher are the 2.5\% and 97.5\% quantiles, 
     respectively.} 
\label{tab:effSummA}
\begin{tabular}{rrrr}
  \hline
Year & Lower & Mean & Higher \\ 
  \hline
1997 & 1.00 & 1.20 & 1.20 \\ 
  1998 & 1.42 & 1.59 & 1.62 \\ 
  1999 & 1.59 & 1.60 & 1.61 \\ 
  2000 & 1.35 & 1.40 & 1.42 \\ 
  2001 & 0.96 & 1.00 & 1.02 \\ 
  2002 & 0.96 & 1.00 & 1.01 \\ 
  2003 & 1.59 & 1.61 & 1.64 \\ 
  2004 & 1.60 & 1.60 & 1.65 \\ 
  2005 & 1.40 & 1.41 & 1.43 \\ 
  2006 & 1.19 & 1.21 & 1.24 \\ 
  2007 & 0.98 & 1.01 & 1.03 \\ 
  2008 & 0.99 & 1.01 & 1.03 \\ 
  2009 & 1.38 & 1.40 & 1.42 \\ 
  2010 & 1.59 & 1.61 & 1.63 \\ 
  2011 & 1.18 & 1.20 & 1.24 \\ 
  2012 & 0.79 & 0.80 & 0.83 \\ 
  2013 & 1.18 & 1.20 & 1.21 \\ 
  2014 & 1.38 & 1.41 & 1.43 \\ 
   \hline
\end{tabular}
\end{table}
% And from iphcSerBallhooksYYR.tex:
% latex table generated in R 3.1.0 by xtable 1.7-3 package
% Fri Apr 17 14:28:12 2015
\begin{table}[p]
\centering
\caption{For series B, summary of effective skate numbers, 
     $E_{it}$, for each
     year. Lower and Higher are the 2.5\% and 97.5\% quantiles, 
     respectively.} 
\label{tab:effSummB}
\begin{tabular}{rrrr}
  \hline
Year & Lower & Mean & Higher \\ 
  \hline
1995 & 4.76 & 4.99 & 5.08 \\ 
  1996 & 4.82 & 4.93 & 5.00 \\ 
  2003 & 7.90 & 7.99 & 8.11 \\ 
  2004 & 7.90 & 7.90 & 8.03 \\ 
  2005 & 6.96 & 7.00 & 7.03 \\ 
  2006 & 5.84 & 5.96 & 6.08 \\ 
  2007 & 4.87 & 4.98 & 5.02 \\ 
  2008 & 4.92 & 4.98 & 5.02 \\ 
  2009 & 6.89 & 6.98 & 7.10 \\ 
  2010 & 7.95 & 8.01 & 8.11 \\ 
  2011 & 5.90 & 5.93 & 6.02 \\ 
  2012 & 3.89 & 4.01 & 4.10 \\ 
  2014 & 6.92 & 7.01 & 7.17 \\ 
   \hline
\end{tabular}
\end{table}

For the overlapping years 2003-2012 and 2014, the mean effective skate numbers
for series A are slightly over 20\% of those for series B. This is because
skates had a mean of just under 100~observed hooks, and so the first 20~hooks in
each skate comprise just over 20\% of the observed hooks. Thus the scaling ratio
$\tilde{H}_{it} /H_{it}$ in (\ref{effSkateScale}) is just over 0.2.
% Checked by doing 
% summary(filter(skateUniqHooksKeep, chumObsHooksPerSkate > 0)$chumObsHooksPerSkate)      in IPHCjoin6.Snw:
%    Min. 1st Qu.  Median    Mean 3rd Qu.    Max. 
%   19.00   98.00  100.00   99.18  100.00  120.00 
The lowest value, for 2012, is due to only four skates (those with Chum Salmon
as bait) being usable for this analysis.

 
% Start of text and figures from iphcCombineAB-YYR.tex:

The resulting bootstrapped catch rate indices for the two Series are shown in
Figure~\ref{fig:bcaKeepSerA}. For Series~A, the highest mean catch rate is at
the start of the series (1997), followed by four constant years, then a drop in
2002. This is unlike the equivalent Series~A for Redbanded Rockfish, which
exhibited an increase in catch rates in the early years (1997-1999), followed
later by a decline in 2002 to a lower level \citep{ehs17}. For Yelloweye, there
is a period of overall increase from 2002-2010, followed by a decline that
results in 2014 having the lowest average catch rate. For Series~B,
Figure~\ref{fig:bcaKeepSerA}(b) shows 1995 and 1996 to be the highest years,
with 2014 the lowest.

Values for the indices for Series~A and Series~B are given in
Tables~\ref{tab:bcaKeepSerA} and \ref{tab:bcaKeepSerB}, respectively, as well as
the number of sets each year and the proportion of sets in each year that did
not catch Yelloweye Rockfish. The early years have slightly fewer sets than the
135 that occurred from 2001 onwards. Year 2008 has only 134 sets because for
station number 2113 the hook-tally sheet was lost overboard
\citep{yfcd11}.  % a wasted afternoon sorting that one out.

\twofig{bcaKeepSerA}{bcaKeepSerB}{Catch rate index (number of individual
  Yelloweye Rockfish caught per skate) for (a) Series A and (b) Series B. For a
  given year, the catch rate for each set is calculated from (\ref{catchPerSet})
  or (\ref{catchPerSet20}) as appropriate. These catch rates are then resampled
  for 10,000 bootstrap values, from which a bootstrapped mean (open circles) and
  95\% bias-corrected and adjusted confidence intervals (bars) are
  calculated. Small black closed circles are sample means (not bootstrapped),
  and essentially equal the bootstrapped means.}



% latex table generated in R 3.1.0 by xtable 1.7-3 package
% Thu Mar 05 16:20:35 2015
\begin{table}[tp]
\centering
\caption{Catch rates by year for Series A.
     `Sample $\bar{I}_t$' is the sample mean. B'ed means bootstrapped 
     value. `No YYR' is the proportion of sets that did not catch \spName~that
     year. Lower and higher are the 
     lower and upper bounds of the 95\% bias-corrected and adjusted (BCa)
     confidence intervals.} 
\label{tab:bcaKeepSerA}
\begin{tabular}{rrrrrrrr}
  \hline
Year & Sets, $n_t$ & No YYR & Sample $\bar{I}_t$ & B'ed $I_t$ & B'ed $I_t$ lower & B'ed $I_t$ higher & B'ed $I_t$ CV \\ 
  \hline
1997 & 121 & 0.66 & 2.31 & 2.31 & 1.58 & 3.37 & 0.19 \\ 
  1998 & 128 & 0.66 & 1.85 & 1.85 & 1.21 & 3.08 & 0.23 \\ 
  1999 & 134 & 0.62 & 1.73 & 1.72 & 1.18 & 2.56 & 0.20 \\ 
  2000 & 129 & 0.64 & 1.75 & 1.75 & 1.21 & 2.51 & 0.18 \\ 
  2001 & 135 & 0.70 & 1.77 & 1.77 & 1.20 & 2.60 & 0.19 \\ 
  2002 & 135 & 0.75 & 0.92 & 0.92 & 0.61 & 1.53 & 0.23 \\ 
  2003 & 135 & 0.67 & 1.07 & 1.06 & 0.72 & 1.69 & 0.22 \\ 
  2004 & 135 & 0.69 & 1.28 & 1.28 & 0.87 & 1.92 & 0.20 \\ 
  2005 & 135 & 0.69 & 1.17 & 1.16 & 0.79 & 1.75 & 0.20 \\ 
  2006 & 135 & 0.76 & 1.16 & 1.16 & 0.74 & 1.80 & 0.22 \\ 
  2007 & 135 & 0.76 & 1.05 & 1.05 & 0.66 & 1.65 & 0.23 \\ 
  2008 & 134 & 0.77 & 1.16 & 1.16 & 0.72 & 1.98 & 0.26 \\ 
  2009 & 135 & 0.71 & 1.45 & 1.45 & 0.95 & 2.24 & 0.22 \\ 
  2010 & 135 & 0.68 & 1.67 & 1.67 & 1.10 & 2.68 & 0.23 \\ 
  2011 & 135 & 0.71 & 1.06 & 1.06 & 0.71 & 1.57 & 0.20 \\ 
  2012 & 135 & 0.77 & 0.88 & 0.88 & 0.57 & 1.45 & 0.24 \\ 
  2013 & 135 & 0.77 & 0.98 & 0.97 & 0.62 & 1.57 & 0.24 \\ 
  2014 & 135 & 0.76 & 0.68 & 0.68 & 0.43 & 1.15 & 0.25 \\ 
   \hline
\end{tabular}
\end{table}% latex table generated in R 3.1.0 by xtable 1.7-3 package
% Thu Mar 05 16:20:35 2015
\begin{table}[tbp]
\centering
\caption{Catch rates by year for Series B.
     `Sample $\bar{I}_t$' is the sample mean. B'ed means bootstrapped 
     value. `No YYR' is the proportion of sets that did not catch \spName~that
     year. Lower and higher are the 
     lower and upper bounds of the 95\% bias-corrected and adjusted (BCa)
     confidence intervals.} 
\label{tab:bcaKeepSerB}
\begin{tabular}{rrrrrrrr}
  \hline
Year & Sets, $n_t$ & No YYR & Sample $\bar{I}_t$ & B'ed $I_t$ & B'ed $I_t$ lower & B'ed $I_t$ higher & B'ed $I_t$ CV \\ 
  \hline
1995 & 115 & 0.70 & 2.17 & 2.17 & 1.37 & 3.53 & 0.24 \\ 
  1996 & 120 & 0.61 & 1.87 & 1.87 & 1.26 & 2.79 & 0.20 \\ 
  2003 & 135 & 0.55 & 1.05 & 1.05 & 0.70 & 1.69 & 0.22 \\ 
  2004 & 135 & 0.59 & 1.24 & 1.24 & 0.86 & 1.80 & 0.19 \\ 
  2005 & 135 & 0.57 & 1.11 & 1.11 & 0.76 & 1.60 & 0.19 \\ 
  2006 & 135 & 0.61 & 1.10 & 1.11 & 0.72 & 1.66 & 0.21 \\ 
  2007 & 135 & 0.67 & 0.90 & 0.90 & 0.60 & 1.34 & 0.21 \\ 
  2008 & 134 & 0.60 & 1.07 & 1.07 & 0.69 & 1.69 & 0.23 \\ 
  2009 & 135 & 0.57 & 1.27 & 1.26 & 0.87 & 1.89 & 0.20 \\ 
  2010 & 135 & 0.59 & 1.49 & 1.49 & 0.99 & 2.36 & 0.22 \\ 
  2011 & 135 & 0.61 & 1.06 & 1.06 & 0.73 & 1.58 & 0.20 \\ 
  2012 & 135 & 0.69 & 0.99 & 0.98 & 0.65 & 1.55 & 0.22 \\ 
  2014 & 135 & 0.62 & 0.66 & 0.66 & 0.44 & 1.03 & 0.22 \\ 
   \hline
\end{tabular}
\end{table}% latex table generated in R 3.1.0 by xtable 1.7-3 package

\clearpage

\subsection{CONSTRUCTING SERIES~AB THAT COVERS ALL 20~YEARS}

We wish to join up the 1995 and 1996 data from Series B (Figure
\ref{fig:bcaKeepSerA}(b)) to the 1997-2014 data from Series A (Figure
\ref{fig:bcaKeepSerA}(a)). The 1995 and 1996 data are only available as numbers
of \spName~caught for all hooks, and not as numbers caught in the first 20 hooks
(Table~\ref{tab:IPHCdata}). For 1997-2002 we only have numbers caught for the
first 20 hooks. But for 2003-2012 and 2014 we have hook-by-hook data, and so can
compute catch rates for all hooks or based on just the first 20 hooks
(i.e.~these overlapping years are the only years that contribute to both
Series~A and Series~B).

For Series A, define $G_A$ to be the geometric mean of the bootstrapped annual
means, with the geometric mean based only on the overlapping years (2003-2012
and 2014). Define $G_B$ similarly for Series B. By dividing the bootstrapped
values for each series by their respective geometric means, we obtain Figure
\ref{fig:bcaKeepSerAB1}(a). This shows that the rescaled Series A and Series B
are very similar for the overlapping years. Thus, on this scale, the 1995 and
1996 values from Series B can be compared to the full Series A data.

We can therefore append the 1995 and 1996 values from the rescaled Series~B in
Figure \ref{fig:bcaKeepSerAB1}(a) to the original Series A values (Figure
\ref{fig:bcaKeepSerA}(a)) by multiplying them by $G_A$, to yield the index
series in Figure \ref{fig:bcaKeepSerAB1}(b) that has units of `numbers per
effective skate'. Equivalently, the original 1995 and 1996 values from
Figure~\ref{fig:bcaKeepSerA}(b) have thus been multiplied by $G_A / G_B$ to give
those in Figure~\ref{fig:bcaKeepSerAB1}(b).

The values for the merged Series AB are given in Table~\ref{tab:serAB}. We next show that Series~AB can be considered as a coastwide index (despite not including the WCVI). % Series~AB is consequently what we use as an input in the population model.

\twofig{bcaKeepSerAB1}{bcaKeepSerAB2}{(a) Each of the two catch rate series from Figure \ref{fig:bcaKeepSerA} is divided by the geometric mean of its bootstrapped annual means (with the geometric mean based on the overlapping years only). (b) The catch rate index Series~AB that could be used as a model input (although the index in Appendix~C was used in this stock assessment). Series~AB extends the original Series~A by incorporating the suitably scaled 1995 and 1996 values from Series~B (see text).}

% Thu Mar 05 16:20:35 2015
\begin{table}[tp]
\centering
\caption{Catch rates by year for Series~AB, constructed by combining 
     1995 and 1996 data
     from Series B with the full data for Series A. The 1995 and 1996 values
     were rescaled by multiplying them by the ratio of the geometric means
     of the bootstrapped means for the two series for the overlapping 
     years, $G_A / G_B$. Values are
     $G_A=$ 1.12  and $G_B$= 1.06 such that $G_A / G_B=$ 1.05 `Sample $\bar{I}_t$' is the sample mean. B'ed means bootstrapped 
     value. `No YYR' is the proportion of sets that did not catch \spName
     ~that year. Lower and higher are the 
     lower and upper bounds of the 95\% bias-corrected and adjusted (BCa)
     confidence intervals.} 
\label{tab:serAB}
\begin{tabular}{rrrrrrrr}
  \hline
Year & Sets, $n_t$ & No YYR & Sample $\bar{I}_t$ & B'ed $I_t$ & B'ed $I_t$ lower & B'ed $I_t$ higher & B'ed $I_t$ CV \\ 
  \hline
1995 & 115 & 0.70 & 2.28 & 2.28 & 1.44 & 3.71 & 0.24 \\ 
  1996 & 120 & 0.61 & 1.96 & 1.97 & 1.32 & 2.93 & 0.20 \\ 
  1997 & 121 & 0.66 & 2.31 & 2.31 & 1.58 & 3.37 & 0.19 \\ 
  1998 & 128 & 0.66 & 1.85 & 1.85 & 1.21 & 3.08 & 0.23 \\ 
  1999 & 134 & 0.62 & 1.73 & 1.72 & 1.18 & 2.56 & 0.20 \\ 
  2000 & 129 & 0.64 & 1.75 & 1.75 & 1.21 & 2.51 & 0.18 \\ 
  2001 & 135 & 0.70 & 1.77 & 1.77 & 1.20 & 2.60 & 0.19 \\ 
  2002 & 135 & 0.75 & 0.92 & 0.92 & 0.61 & 1.53 & 0.23 \\ 
  2003 & 135 & 0.67 & 1.07 & 1.06 & 0.72 & 1.69 & 0.22 \\ 
  2004 & 135 & 0.69 & 1.28 & 1.28 & 0.87 & 1.92 & 0.20 \\ 
  2005 & 135 & 0.69 & 1.17 & 1.16 & 0.79 & 1.75 & 0.20 \\ 
  2006 & 135 & 0.76 & 1.16 & 1.16 & 0.74 & 1.80 & 0.22 \\ 
  2007 & 135 & 0.76 & 1.05 & 1.05 & 0.66 & 1.65 & 0.23 \\ 
  2008 & 134 & 0.77 & 1.16 & 1.16 & 0.72 & 1.98 & 0.26 \\ 
  2009 & 135 & 0.71 & 1.45 & 1.45 & 0.95 & 2.24 & 0.22 \\ 
  2010 & 135 & 0.68 & 1.67 & 1.67 & 1.10 & 2.68 & 0.23 \\ 
  2011 & 135 & 0.71 & 1.06 & 1.06 & 0.71 & 1.57 & 0.20 \\ 
  2012 & 135 & 0.77 & 0.88 & 0.88 & 0.57 & 1.45 & 0.24 \\ 
  2013 & 135 & 0.77 & 0.98 & 0.97 & 0.62 & 1.57 & 0.24 \\ 
  2014 & 135 & 0.76 & 0.68 & 0.68 & 0.43 & 1.15 & 0.25 \\ 
   \hline
\end{tabular}
\end{table}


% End of text and figures from iphcCombineAB-YYR.tex


% ****** Series D and Series A *****

\subsection{CONSTRUCTING SERIES~D (20 HOOKS, COASTWIDE) AND COMPARING IT WITH SERIES~A (20 HOOKS, NORTH OF VANCOUVER ISLAND)}\label{sec:compareAD}

% From CST-YYR/seriesD-YYR/compareA-D.tex, pretty much straight in and taking out the first figure:

We now construct Series~D, which is for the first 20~hooks of each skate (like for Series~A) but covers the whole coast, including the WCVI (unlike Series~A), as was summarised in Table~\ref{tab:seriesSumm}. We then show that Series~A (north of Vancouver Island (VI)) and Series~D (whole coast) show similar relative changes over the overlapping years, and so Series~A can be considered representative of the whole coast, i.e.~the population off the WCVI is not showing a different relative trend to the rest of the coast. % For B and C it turns out the absolute catch rates were different, but the relative year-to-year changes looked the same.

Figure~\ref{fig:CST-YYR/seriesD-YYR/bcaKeepSerAD1}(a) shows the absolute catch
rate index for Series~A (first 20~hooks from each skate, north of VI, as in
Figure \ref{fig:bcaKeepSerA}(a)), together with the shorter time series for
Series~D (first 20~hooks, coastwide, for 1999 and 2001-2014). For all
overlapping years, the Series~D means and confidence intervals are less than
those for Series~A.

For Series~A, define $G'_{A}$ to be the geometric mean of the bootstrapped
annual means, with the geometric mean based only the years that overlap with
Series~D (i.e.~1999 and 2001-2014). Define $G_D$ similarly for Series~D. By
dividing the bootstrapped values for each series by their respective geometric
means, we obtain Figure~\ref{fig:CST-YYR/seriesD-YYR/bcaKeepSerAD1}(b). The
rescaled means are very close to each other, such that for the overlapping years
the temporal patterns for Series~A and Series~D are very similar. (Values for
Series A and Series D are given in Tables \ref{tab:bcaKeepSerA}
and~\ref{tab:bcaKeepSerD}, respectively, and for the rescaled series in
Tables~\ref{tab:serAscaled} and~\ref{tab:serDscaled}).
% Tue Apr 21 10:26:48 2015
\begin{table}[tp]
\centering
\caption{Catch rates by year for Series~D, which is first 20~hooks
      for all stations (coastwide) for 1999 and 2001-2014.
     `Sample $\bar{I}_t$' is the sample mean. B'ed means bootstrapped 
     value. `No YYR' is the proportion of sets that did not catch \spName~that
     year. Lower and higher are the 
     lower and upper bounds of the 95\% bias-corrected and adjusted (BCa)
     confidence intervals.} 
\label{tab:bcaKeepSerD}
\begin{tabular}{rrrrrrrr}
  \hline
Year & Sets, $n_t$ & No YYR & Sample $\bar{I}_t$ & B'ed $I_t$ & B'ed $I_t$ lower & B'ed $I_t$ higher & B'ed $I_t$ CV \\ 
  \hline
1999 & 168 & 0.64 & 1.55 & 1.55 & 1.09 & 2.22 & 0.18 \\ 
  2001 & 170 & 0.71 & 1.64 & 1.64 & 1.16 & 2.32 & 0.18 \\ 
  2002 & 170 & 0.76 & 0.80 & 0.80 & 0.54 & 1.27 & 0.21 \\ 
  2003 & 170 & 0.70 & 0.91 & 0.91 & 0.63 & 1.42 & 0.20 \\ 
  2004 & 170 & 0.70 & 1.15 & 1.15 & 0.80 & 1.70 & 0.19 \\ 
  2005 & 170 & 0.70 & 1.02 & 1.02 & 0.72 & 1.50 & 0.19 \\ 
  2006 & 170 & 0.77 & 0.99 & 0.99 & 0.66 & 1.51 & 0.21 \\ 
  2007 & 170 & 0.78 & 0.92 & 0.92 & 0.61 & 1.40 & 0.21 \\ 
  2008 & 169 & 0.77 & 1.08 & 1.08 & 0.72 & 1.71 & 0.22 \\ 
  2009 & 170 & 0.71 & 1.33 & 1.33 & 0.93 & 1.98 & 0.19 \\ 
  2010 & 170 & 0.71 & 1.41 & 1.41 & 0.95 & 2.20 & 0.22 \\ 
  2011 & 170 & 0.71 & 0.98 & 0.98 & 0.69 & 1.39 & 0.18 \\ 
  2012 & 170 & 0.78 & 0.80 & 0.81 & 0.53 & 1.24 & 0.22 \\ 
  2013 & 170 & 0.79 & 0.85 & 0.85 & 0.56 & 1.33 & 0.22 \\ 
  2014 & 170 & 0.77 & 0.62 & 0.62 & 0.40 & 0.97 & 0.22 \\ 
   \hline
\end{tabular}
\end{table}% latex table generated in R 3.1.0 by xtable 1.7-3 package

Thus the relative patterns for Series~A and Series~D appear similar for the
overlapping years. But the absolute catch rates in
Figure~\ref{fig:CST-YYR/seriesD-YYR/bcaKeepSerAD1}(a) show that inclusion of the
WCVI stations in Series~D consistently reduces the catch rates from those of
Series~A (that did not include the WCVI stations). So while inclusion of the
WCVI stations does not appear to change the \emph{relative} pattern of the index
of the population, it does change the absolute values. Therefore the stations
off the WCVI have to be included or excluded consistently to construct an index
series; since we have more years that do not have stations off the WCVI
(Table~\ref{tab:seriesSumm}), we consistently exclude these stations (giving
Series~A).

The stations off the WCVI have lower average catch rates than the remaining
stations. Excluding them in Series~A increases the (geometric and arithmetic)
means of the catch rates by 12\% (Table~\ref{tab:serDscaled}) compared to
Series~D. Thus, in Figure~\ref{fig:CST-YYR/seriesD-YYR/bcaKeepSerAD1}(a) we
cannot simply join up the 1997, 1998 and 1999 Series~A values with the Series~D
values for the other years, because the 1997, 1998 and 1999 values exclude
stations off the WCVI that appear to have lower catch rates.
% (at least for the later years that we have data for)

So the population off the WCVI appears to be changing in the same way as the
rest of the coast for the overlapping years (1999, 2001-2014), it just has lower
catch rates in the IPHC survey than for the rest of the coast. Thus, Series~A
can be considered to be an index of the coastwide population for the overlapping
years.


% Did have this after 'input to the assessment model.' but I think will just create confusion:
% The model actually uses this as a relative (not absolute) index because it scales it by a catchability parameter -- the fact that we had to rescale Series~AB and Series~C in Figure~\ref{fig:bcaKeepSerAD1}(b) to show that they have the same pattern does not preclude using the absolute Series~AB in the model. The absolute series is preferable because it has interpretable units (numbers of \spName~caught per effective skate).

\twofig{CST-YYR/seriesD-YYR/bcaKeepSerAD1}{CST-YYR/seriesD-YYR/bcaKeepSerAD2}{(a)
  Catch rate index (number of individual Yelloweye Rockfish caught per effective
  skate) for Series~A (20~hooks, north of Vancouver Island) and Series~D
  (20~hooks, coastwide) for 1999 and 2001-2014 (plus 1997, 1998 and 2000 for
  Series~A). For a given year, the catch rate for each set is calculated from
  (\ref{catchPerSet20}). These catch rates are then resampled for 10,000
  bootstrap values, from which a bootstrapped mean (open circles) and 95\%
  bias-corrected and adjusted confidence intervals (bars) are calculated. (b)
  Each series is divided by the geometric mean of its bootstrapped annual means
  (with the geometric mean based on the overlapping years only), to enable
  comparison in the overlapping years.}


% Tue Apr 21 10:26:48 2015
\begin{table}[tp]
\centering
\caption{As for Table \ref{tab:bcaKeepSerA} for Series A, 
     but with catch rates rescaled by dividing
     by $G'_{A}=$ 1.16 , 
     the geometric mean of the bootstrapped means for the years that overlap
     with Series D, for plotting in 
     Figure~\ref{fig:CST-YYR/seriesD-YYR/bcaKeepSerAD1}(b).
     `Sample $\bar{I}_t$' is the rescaled sample mean. B'ed means bootstrapped 
     value. `No YYR' is the proportion of sets that did not catch \spName~that
     year. Lower and higher are the 
     lower and upper bounds of the rescaled 
     95\% bias-corrected and adjusted (BCa)
     confidence intervals.} 
\label{tab:serAscaled}
\begin{tabular}{rrrrrrrr}
  \hline
Year & Sets, $n_t$ & No YYR & Sample $\bar{I}_t$ & B'ed $I_t$ & B'ed $I_t$ lower & B'ed $I_t$ higher & B'ed $I_t$ CV \\ 
  \hline
1997 & 121 & 0.66 & 1.99 & 1.99 & 1.36 & 2.90 & 0.19 \\ 
  1998 & 128 & 0.66 & 1.59 & 1.60 & 1.04 & 2.65 & 0.23 \\ 
  1999 & 134 & 0.62 & 1.49 & 1.49 & 1.01 & 2.20 & 0.20 \\ 
  2000 & 129 & 0.64 & 1.50 & 1.51 & 1.04 & 2.16 & 0.18 \\ 
  2001 & 135 & 0.70 & 1.53 & 1.53 & 1.04 & 2.24 & 0.19 \\ 
  2002 & 135 & 0.75 & 0.79 & 0.79 & 0.52 & 1.32 & 0.23 \\ 
  2003 & 135 & 0.67 & 0.92 & 0.92 & 0.62 & 1.46 & 0.22 \\ 
  2004 & 135 & 0.69 & 1.10 & 1.11 & 0.75 & 1.65 & 0.20 \\ 
  2005 & 135 & 0.69 & 1.01 & 1.00 & 0.68 & 1.51 & 0.20 \\ 
  2006 & 135 & 0.76 & 1.00 & 1.00 & 0.64 & 1.55 & 0.22 \\ 
  2007 & 135 & 0.76 & 0.90 & 0.90 & 0.57 & 1.42 & 0.23 \\ 
  2008 & 134 & 0.77 & 1.00 & 1.00 & 0.62 & 1.70 & 0.26 \\ 
  2009 & 135 & 0.71 & 1.25 & 1.25 & 0.82 & 1.93 & 0.22 \\ 
  2010 & 135 & 0.68 & 1.44 & 1.44 & 0.94 & 2.31 & 0.23 \\ 
  2011 & 135 & 0.71 & 0.92 & 0.91 & 0.61 & 1.36 & 0.20 \\ 
  2012 & 135 & 0.77 & 0.76 & 0.76 & 0.49 & 1.25 & 0.24 \\ 
  2013 & 135 & 0.77 & 0.84 & 0.84 & 0.53 & 1.35 & 0.24 \\ 
  2014 & 135 & 0.76 & 0.58 & 0.59 & 0.37 & 0.99 & 0.25 \\ 
   \hline
\end{tabular}
\end{table}% latex table generated in R 3.1.0 by xtable 1.7-3 package
% Tue Apr 21 10:26:48 2015
\begin{table}[tp]
\centering
\caption{As for Table \ref{tab:bcaKeepSerD} for Series D, 
     but with catch rates rescaled by dividing
     by $G_{D}=$ 1.03 , 
     the geometric mean of the bootstrapped means for the years that overlap
     with Series A, for plotting in 
     Figure~\ref{fig:CST-YYR/seriesD-YYR/bcaKeepSerAD1}(b).
     `Sample $\bar{I}_t$' is the rescaled sample mean. B'ed means bootstrapped 
     value. `No YYR' is the proportion of sets that did not catch \spName~that
     year. Lower and higher are the 
     lower and upper bounds of the rescaled 
     95\% bias-corrected and adjusted (BCa)
     confidence intervals. The ratio of the geometric means is
     $G'_{A}/G_{D}=$ 1.12 ; the respective arithmetic means 
     are 1.20 and 1.07, with a ratio of  1.12.} 
\label{tab:serDscaled}
\begin{tabular}{rrrrrrrr}
  \hline
Year & Sets, $n_t$ & No YYR & Sample $\bar{I}_t$ & B'ed $I_t$ & B'ed $I_t$ lower & B'ed $I_t$ higher & B'ed $I_t$ CV \\ 
  \hline
1999 & 168 & 0.64 & 1.50 & 1.50 & 1.06 & 2.15 & 0.18 \\ 
  2001 & 170 & 0.71 & 1.58 & 1.58 & 1.12 & 2.25 & 0.18 \\ 
  2002 & 170 & 0.76 & 0.77 & 0.77 & 0.52 & 1.23 & 0.21 \\ 
  2003 & 170 & 0.70 & 0.88 & 0.88 & 0.60 & 1.37 & 0.20 \\ 
  2004 & 170 & 0.70 & 1.11 & 1.11 & 0.78 & 1.64 & 0.19 \\ 
  2005 & 170 & 0.70 & 0.99 & 0.99 & 0.70 & 1.45 & 0.19 \\ 
  2006 & 170 & 0.77 & 0.96 & 0.95 & 0.64 & 1.46 & 0.21 \\ 
  2007 & 170 & 0.78 & 0.89 & 0.89 & 0.59 & 1.36 & 0.21 \\ 
  2008 & 169 & 0.77 & 1.04 & 1.04 & 0.69 & 1.66 & 0.22 \\ 
  2009 & 170 & 0.71 & 1.28 & 1.29 & 0.90 & 1.92 & 0.19 \\ 
  2010 & 170 & 0.71 & 1.37 & 1.36 & 0.92 & 2.13 & 0.22 \\ 
  2011 & 170 & 0.71 & 0.95 & 0.95 & 0.67 & 1.35 & 0.18 \\ 
  2012 & 170 & 0.78 & 0.78 & 0.78 & 0.52 & 1.20 & 0.22 \\ 
  2013 & 170 & 0.79 & 0.82 & 0.82 & 0.54 & 1.29 & 0.22 \\ 
  2014 & 170 & 0.77 & 0.60 & 0.60 & 0.39 & 0.94 & 0.22 \\ 
   \hline
\end{tabular}
\end{table}

% End of from compareA-D.tex


\clearpage

% ****** Series B and Series C *****

\subsection{CONSTRUCTING SERIES~C (ALL HOOKS, COASTWIDE) AND COMPARING IT WITH SERIES~B (ALL HOOKS, NORTH OF VANCOUVER ISLAND)}\label{sec:compareBC}

% From CST-YYR/compareB-C.tex, pretty much straight in and taking out the first figure, and adding in directory name for figures.

We now construct Series~C, which is for all hooks from each skate (like for
Series~B) but covers the whole coast, including the WCVI (unlike Series~B), as
was summarised in Table~\ref{tab:seriesSumm}. We then show that Series~B (north
of VI) and Series~C (whole coast) are similar over the overlapping years, and so
Series~B can be considered representative of the whole coast, i.e.~the
population off the WCVI is not showing a different relative trend to the rest of
the coast, as demonstrated above for Series~A
and~D. % For B and C it turns out the absolute catch rates were different, but the relative year-to-year changes looked the same.

Figure~\ref{fig:CST-YYR/bcaKeepSerBC1}(a) shows the absolute catch rate index
for Series~B (all hooks, north of VI), together with the shorter time series for
Series C (all hooks, coastwide, for 2003-2012 and 2014). For all overlapping
years, the Series~C means and confidence intervals are less than those for
Series~B.

For Series~B, define $G'_{B}$ to be the geometric mean of the bootstrapped
annual means, with the geometric mean based only the years that overlap with
Series~C (i.e.~2003 to 2012 and 2014). Define $G_C$ similarly for Series C. By
dividing the bootstrapped values for each series by their respective geometric
means, we obtain Figure~\ref{fig:CST-YYR/bcaKeepSerBC1}(b). The rescaled means
are very close to each other, such that for the overlapping years the temporal
patterns for Series~B and Series~C are very similar. (Values for Series B and
Series C are given in Tables \ref{tab:bcaKeepSerB} and~\ref{tab:bcaKeepSerC},
respectively, and for the rescaled series in Tables~\ref{tab:serBscaled}
and~\ref{tab:serCscaled}).

% Tue Apr 21 11:15:46 2015
\begin{table}[bp]    % b here since new subsection
\centering
\caption{Catch rates by year for Series C, which is all hooks for 
      all stations (coastwide) from 2003-2012 and 2014.
     `Sample $\bar{I}_t$' is the sample mean. B'ed means bootstrapped 
     value. `No YYR' is the proportion of sets that did not catch \spName~that
     year. Lower and higher are the 
     lower and upper bounds of the 95\% bias-corrected and adjusted (BCa)
     confidence intervals.} 
\label{tab:bcaKeepSerC}
\begin{tabular}{rrrrrrrr}
  \hline
Year & Sets, $n_t$ & No YYR & Sample $\bar{I}_t$ & B'ed $I_t$ & B'ed $I_t$ lower & B'ed $I_t$ higher & B'ed $I_t$ CV \\ 
  \hline
2003 & 170 & 0.58 & 0.90 & 0.90 & 0.62 & 1.43 & 0.21 \\ 
  2004 & 170 & 0.59 & 1.13 & 1.13 & 0.81 & 1.60 & 0.17 \\ 
  2005 & 170 & 0.59 & 1.00 & 1.00 & 0.71 & 1.40 & 0.17 \\ 
  2006 & 170 & 0.61 & 0.99 & 0.99 & 0.69 & 1.44 & 0.19 \\ 
  2007 & 170 & 0.67 & 0.82 & 0.82 & 0.56 & 1.18 & 0.19 \\ 
  2008 & 169 & 0.62 & 1.00 & 1.00 & 0.68 & 1.49 & 0.20 \\ 
  2009 & 170 & 0.56 & 1.18 & 1.17 & 0.85 & 1.69 & 0.18 \\ 
  2010 & 170 & 0.61 & 1.28 & 1.28 & 0.88 & 1.98 & 0.21 \\ 
  2011 & 170 & 0.62 & 0.94 & 0.94 & 0.66 & 1.37 & 0.19 \\ 
  2012 & 170 & 0.71 & 0.89 & 0.89 & 0.61 & 1.34 & 0.20 \\ 
  2014 & 170 & 0.64 & 0.60 & 0.60 & 0.42 & 0.89 & 0.19 \\ 
   \hline
\end{tabular}
\end{table}% latex table generated in R 3.1.0 by xtable 1.7-3 package


Thus the relative patterns for Series~B and Series~C appear similar for the
overlapping years. But the absolute catch rates in
Figure~\ref{fig:CST-YYR/bcaKeepSerBC1}(a) show that inclusion of the WCVI
stations in Series~C consistently reduces the catch rates from those of Series~B
(that did not include the WCVI stations). So while inclusion of the WCVI
stations does not appear to change the \emph{relative} pattern of the index of
the population, it does change the absolute values. Therefore the stations off
the WCVI have to be included or excluded consistently to construct an index
series; since we have more years that do not have stations off the WCVI
(Table~\ref{tab:seriesSumm}), we consistently exclude these stations (giving
Series~B).

The stations off the WCVI have lower average catch rates than the remaining
stations. Excluding them in Series~B increases the (geometric and arithmetic)
means of the catch rates by 11\% (Table~\ref{tab:serCscaled}) compared to
Series~C. Thus, in Figure~\ref{fig:CST-YYR/bcaKeepSerBC1}(a) we cannot simply
join up the 1995 and 1996 Series~B values with the Series~C values for the other
years, because the 1995 and 1996 values exclude stations off the WCVI that
appear (at least for the later years that we have data for) to have lower catch
rates.

So the population off the WCVI appears to be changing in the same way as the
rest of the coast for the overlapping years (2003-2012 and 2014), it just has
lower catch rates in the IPHC survey than for the rest of the coast. Thus,
Series~B can be considered to be an index of the coastwide population for the
overlapping years.

\twofig{CST-YYR/bcaKeepSerBC1}{CST-YYR/bcaKeepSerBC2}{(a) Catch rate index
  (number of individual Yelloweye Rockfish caught per effective skate) for
  Series~B (all hooks, north of Vancouver Island) and Series~C (all hooks,
  coastwide) from 2003-2012 and 2014 (plus 1995 and 1996 for Series~B). For a
  given year, the catch rate for each set is calculated from
  (\ref{catchPerSet}). These catch rates are then resampled for 10,000 bootstrap
  values, from which a bootstrapped mean (open circles) and 95\% bias-corrected
  and adjusted confidence intervals (bars) are calculated. (b) Each series is
  divided by the geometric mean of its bootstrapped annual means (with the
  geometric mean based on the overlapping years only), to enable comparison in
  the overlapping years.}

% Tue Apr 21 11:15:46 2015
\begin{table}[tp]
\centering
\caption{As for Table \ref{tab:bcaKeepSerB} for Series B, 
     but with catch rates rescaled by dividing
     by $G'_{B}=$ 1.06 , 
     the geometric mean of the bootstrapped means for the years that overlap
     with Series C, for plotting in Figure~\ref{fig:CST-YYR/bcaKeepSerBC1}(b).
     `Sample $\bar{I}_t$' is the rescaled sample mean. B'ed means bootstrapped 
     value. `No YYR' is the proportion of sets that did not catch \spName~that
     year. Lower and higher are the 
     lower and upper bounds of the rescaled 
     95\% bias-corrected and adjusted (BCa)
     confidence intervals.} 
\label{tab:serBscaled}
\begin{tabular}{rrrrrrrr}
  \hline
Year & Sets, $n_t$ & No YYR & Sample $\bar{I}_t$ & B'ed $I_t$ & B'ed $I_t$ lower & B'ed $I_t$ higher & B'ed $I_t$ CV \\ 
  \hline
1995 & 115 & 0.70 & 2.04 & 2.04 & 1.29 & 3.32 & 0.24 \\ 
  1996 & 120 & 0.61 & 1.76 & 1.76 & 1.18 & 2.62 & 0.20 \\ 
  2003 & 135 & 0.55 & 0.98 & 0.98 & 0.66 & 1.59 & 0.22 \\ 
  2004 & 135 & 0.59 & 1.16 & 1.16 & 0.81 & 1.69 & 0.19 \\ 
  2005 & 135 & 0.57 & 1.04 & 1.04 & 0.72 & 1.51 & 0.19 \\ 
  2006 & 135 & 0.61 & 1.04 & 1.04 & 0.68 & 1.56 & 0.21 \\ 
  2007 & 135 & 0.67 & 0.84 & 0.84 & 0.56 & 1.26 & 0.21 \\ 
  2008 & 134 & 0.60 & 1.01 & 1.01 & 0.65 & 1.59 & 0.23 \\ 
  2009 & 135 & 0.57 & 1.19 & 1.19 & 0.82 & 1.78 & 0.20 \\ 
  2010 & 135 & 0.59 & 1.40 & 1.40 & 0.93 & 2.21 & 0.22 \\ 
  2011 & 135 & 0.61 & 1.00 & 0.99 & 0.68 & 1.49 & 0.20 \\ 
  2012 & 135 & 0.69 & 0.93 & 0.92 & 0.61 & 1.45 & 0.22 \\ 
  2014 & 135 & 0.62 & 0.62 & 0.62 & 0.41 & 0.97 & 0.22 \\ 
   \hline
\end{tabular}
\end{table}% latex table generated in R 3.1.0 by xtable 1.7-3 package
% Tue Apr 21 11:15:46 2015
\begin{table}[tp]
\centering
\caption{As for Table \ref{tab:bcaKeepSerC} for Series C, 
     but with catch rates rescaled by dividing
     by $G_{C}=$ 0.96 , 
     the geometric mean of the bootstrapped means for the years that overlap
     with Series B, for plotting in Figure~\ref{fig:CST-YYR/bcaKeepSerBC1}(b).
     `Sample $\bar{I}_t$' is the rescaled sample mean. B'ed means bootstrapped 
     value. `No YYR' is the proportion of sets that did not catch \spName~that
     year. Lower and higher are the 
     lower and upper bounds of the rescaled 
     95\% bias-corrected and adjusted (BCa)
     confidence intervals. The ratio of the geometric means is $G'_{B}/G_{C}=$ 1.11 ; the respective arithmetic means are  1.08  and 
      0.98 , with a ratio of  1.11 .} 
\label{tab:serCscaled}
\begin{tabular}{rrrrrrrr}
  \hline
Year & Sets, $n_t$ & No YYR & Sample $\bar{I}_t$ & B'ed $I_t$ & B'ed $I_t$ lower & B'ed $I_t$ higher & B'ed $I_t$ CV \\ 
  \hline
2003 & 170 & 0.58 & 0.94 & 0.94 & 0.64 & 1.49 & 0.21 \\ 
  2004 & 170 & 0.59 & 1.18 & 1.18 & 0.85 & 1.67 & 0.17 \\ 
  2005 & 170 & 0.59 & 1.05 & 1.05 & 0.74 & 1.46 & 0.17 \\ 
  2006 & 170 & 0.61 & 1.04 & 1.04 & 0.72 & 1.51 & 0.19 \\ 
  2007 & 170 & 0.67 & 0.86 & 0.86 & 0.58 & 1.23 & 0.19 \\ 
  2008 & 169 & 0.62 & 1.04 & 1.04 & 0.70 & 1.55 & 0.20 \\ 
  2009 & 170 & 0.56 & 1.23 & 1.22 & 0.88 & 1.76 & 0.18 \\ 
  2010 & 170 & 0.61 & 1.34 & 1.33 & 0.92 & 2.07 & 0.21 \\ 
  2011 & 170 & 0.62 & 0.98 & 0.98 & 0.69 & 1.43 & 0.19 \\ 
  2012 & 170 & 0.71 & 0.93 & 0.93 & 0.63 & 1.40 & 0.20 \\ 
  2014 & 170 & 0.64 & 0.63 & 0.62 & 0.44 & 0.93 & 0.19 \\ 
   \hline
\end{tabular}
\end{table}

% End of from CST-YYR/compareB-C.tex, pretty much straight in and taking out the first figure.

\clearpage

\section{CONCLUSION -- SERIES~AB IS THE LONGEST IPHC SERIES WE CAN CONSTRUCT AND CAN BE CONSIDERED A COASTWIDE INDEX}

For all years that comparisons could be made, the two Series that exclude
stations off the WCVI (Series~A and~B) demonstrated the same relative temporal
patterns as the two Series that included the stations off the WCVI (Series~D
and~B, respectively), and so we assume that the same holds for the few years for
which we are unable to make such a comparison (namely 1995-1998 and 2000). Since
Series~AB is the longest that we are able to construct it would be a suitable
index for the assessment model, and we consider it to be an index of the
coastwide population for all years.

We would therefore recommend that the absolute catch rate index for Series AB in
Figure~\ref{fig:bcaKeepSerAB1}(b), with values in Table~\ref{tab:serAB}, could
be used an input to an assessment model. But for this stock assessment of
\spName~the index from Appendix~C was used.

We note that Series~AB does show a somewhat abrupt drop in catch rates between
2001 and 2002 (a 48\% drop in the sample mean from 1.77 to 0.92). The survey
report for 2002 \citep{dvr03} suggests no methodological reason for such a
decline, since `survey station design and most sampling protocols were left
unchanged from 2001'. Indeed, when comparing the design and sampling protocols
with those from 2001 \citep{dvr02}, we see no alterations that would lead to a
drop in catch rates of \spName. For Pacific Halibut, Figure~2 of \citet{dvr03}
only shows an 8\% drop in catch rates (pounds per skate) from 2001 to 2002 (and
a 6\% increase from 1999 to 2002), further suggesting that there is not a
methodological reason for the survey to have lower catch rates (in general) in
2002 compared to 2001.

\section{DISCUSSION}
% \section{From IPHChookanalysis.tex:}

Future uses of the IPHC survey data for Canadian Pacific groundfish stock
assessments could investigate the use of, for example, generalised linear
models. There was not time to do so for this assessment, but the extensive data
processing and associated R code developed for this work should prove useful for
such work.

Also, further methods could be considered, such as the delta-gamma method
(e.g. \citealt{lbaebp13}) that would explicitly model the zero catch rates seen
for some sets. For the Redbanded Rockfish assessment, it was not clear \emph{a
  priori} whether or not the delta-gamma method would lead to reduced
coefficients of variation for the catch rates (Jean-Baptiste Lecomte, Pacific
Biological Station, Nanaimo, BC, pers.~comm.). The method used here does not
explicitly account for the expected increased number of sets with zero catches
in 2012 compared to other years (recall that the bait experiment in 2012 meant
that only four skates could be used from each set). However, 2012 did not have
the highest proportion of zero catches (Table~\ref{tab:serAB}) and does not look
anomalous (Figure \ref{fig:bcaKeepSerAB1}), and so the methods used here appear
to be suitable.


\clearpage
